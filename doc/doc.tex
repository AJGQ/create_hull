\documentclass[11pt]{article}

\title{Algoritmo para gerar invólucros convexos}
\author{Artur Queiroz - PG38014\\Luís Albuquerque - PG38015}
\begin{document}
\maketitle

\section{Introdução}
Neste trabalho abordamos um dos principais temas de Geometria 
Computacional, Invólucros Convexos.
Estes são muito usados porque ...

Existem várias formas de construir um, mas neste trabalho vamos
nos cingir a implementar o "merge-hull"
Que é um algoritmo que se usa, umas das tecnicas mais importantes
na computação, que se chama "Dividir para conquistar".
Que se baseia em dividir um problema complexo, em problemas mais 
pequenos e mais faceis.

\section{Descrição}
Para implementarmos o algoritmo apenas temos que seguir os passos
a baixo.

\begin{itemize}
    \item Ordenar os pontos por ordem lexicográfica.
    \item Separar os pontos em dois conjuntos A e B, onde A contem os pontos da esquerda e B os da direita.
    \item Calcular o invólucro convexo de A, $\mathcal{A}$ = I(A) e o de B, $\mathcal{B}$ = I(B) recursivamente
    \item No final juntar $\mathcal{A}$ e $\mathcal{B}$, calculando o invólucro convexo de A $\cup$ B.
\end{itemize}

Agora vamos explicar com mais detalhe todo o processo.
\subsection{ Ordenar os Pontos }
Para ordenar os pontos, pode ser usado qualquer algoritmo de ordenação,
tendo em atenção que a escolha do algoritmo de ordenação, pode alterar a
complexidade do algoritmo como um todo.
Nós optamos por escolher o algoritmo de ordenação \textit{merge sort},
além de ter uma das melhores complexidades $\Theta(n\log{}n)$, achamos 
que se enquadra perfeitamente no espirito do algoritmo, "Dividir para conquistar".

\subsection{Descrição de MergeSort}
Input: array, indice esquerdo, indice direito\\

Começando com o indice esquerdo a 0, e a indice direito a (\textit{tamanho do array}) - 1
\begin{itemize}
    \item Primeiro encontra-se o indice médio do Array e divide-se em dois. ( meio = (esquerda + direita)/2 )
    \item Calcular o MergeSort(array, esquerda, meio), com a lista que fica à esquerda 
    \item Calcular o MergeSort(array, meio+1, direita), com a lista que fica à direita
    \item No final junta os dois de forma ordenada.
\end{itemize}

\subsection{ Separar os Pontos em dois conjuntos}

\subsection{ Calcular o invólucro convexo } 

\subsection{ Juntar os invólucros convexos } 


\section{Correção}
Depois de mostrarmos como é o algoritmo,
aqui vamos provar, porque é que o algoritmo faz o que diz que faz.\\

\vspace{.5cm}

Qed.

\section{Complexidade}
A nossa implementação não foi exatamente igual ao algoritmo original,
apesar de não alterar na conta da complexidade assintoticamente.
Por isso vamos avaliar a correção da nossa implementação, e quando 
achamos pertinente, vamos fazer a ressalva, mensionando as diferenças 
em relação ao algoritmo original.

\begin{itemize}
    \item Ordenar os pontos pela cordenada x, tem Complexidade $\Theta(n\log{}n)$
    \item Separar os pontos em dois conjuntos A e B, onde A contém os pontos da esquerda e B os da direita.
    \item Calcular o invólucro convexo de A, $\mathcal{A}$ = I(A) e o de B, $\mathcal{B}$ = I(B) recursivamente
    \item No final juntar $\mathcal{A}$ e $\mathcal{B}$, calculando o invólucro convexo de A $\cup$ B.
\end{itemize}

\section{Conclusão}

%\section{section}
%\subsection*{subsection without number}
%text \textbf{bold text} text. Some math: $2+2=5$
%\subsection{subsection}
%text \emph{emphasized text} text. 
%discovered the structure of DNA.
%
%A table:
%\begin{table}[!th]
%\begin{tabular}{|l|c|r|}
%\hline
%first  &  row  &  data \\
%second &  row  &  data \\
%\hline
%
%\end{tabular}
%\caption{This is the caption}
%\label{ex:table}
%\end{table}
%
%The table is numbered \ref{ex:table}.
\end{document}
