\documentclass[11pt]{article}
\usepackage{amssymb}
\usepackage{amsmath}
\usepackage[portuguese]{babel}
%\usepackage[all]{xy}

\title{Algoritmo para gerar invólucros convexos}
\author{Artur Queiroz - PG38014\\Luís Albuquerque - PG38015}
\begin{document}
\maketitle

\section{Introdução}
Neste trabalho abordamos um dos principais temas de Geometria 
Computacional, Invólucros Convexos.
Estes são muito usados porque têm aplicações em bastantes áreas
como reconhecimento de padrões, processamento de imagem, 
estatística, etc.

Existem várias formas de construir um, mas neste trabalho vamos
nos cingir a implementar o "\textit{merge-hull}"
que é um algoritmo que se usa, umas das técnicas mais importantes
na computação, que se chama "Dividir para conquistar".
Baseia-se em dividir um problema complexo, em problemas mais 
pequenos e mais faceis.

\subsection{Notações utilizadas}
$<\forall x : p(x) : q(x)>$ é equivalente a $\forall_{p(x)} . q(x)$\\ \\
Quando fazemos\\
$\mathcal{R}\{$Descrição$\}$\\
significa que a "Descrição" é a razão pela qual a relação $\mathcal{R}$
é verdade.\\ \\
left(l, p) : "o ponto p está à esquerda da linha l"\\
leftOn(l, p) : "o ponto p está à esquerda ou na linha l"\\
right(l, p) : "o ponto p está à direita da linha l"\\
rightOn(l, p) : "o ponto p está à direita ou na linha l"\\


%--------------------------------------------
%\clearpage
\section{Descrição}
Para implementarmos o algoritmo apenas temos que seguir os passos
a baixo.

\begin{itemize}
    \item Ordenar os pontos por ordem lexicográfica.
    \item Remover os pontos repetidos.
    \item Criar o Invólucro convexo
    \begin{itemize}
        \item Se o conjunto tem apenas 1 elemento, criar um "polígono"
            de 1 elemento e acaba-se a função.
        \item Separar os pontos em dois conjuntos A e B, 
            onde A contém os pontos da esquerda e B os da direita.
        \item Calcular o invólucro convexo de A, 
            $\mathcal{A}$ = I(A) e o de B, $\mathcal{B}$ = I(B) recursivamente
        \item Encontrar os pontos de junção (tangentes).
        \item Juntar $\mathcal{A}$ e $\mathcal{B}$, 
            calculando o invólucro convexo de A $\cup$ B.
    \end{itemize}
\end{itemize}

Agora vamos explicar com mais detalhe todo o processo.
\subsection{ Ordenar os Pontos }
Para ordenar os pontos, pode ser usado qualquer algoritmo de ordenação,
tendo em atenção que a escolha do algoritmo de ordenação, pode alterar a
complexidade do algoritmo como um todo.
Nós optamos por escolher o algoritmo de ordenação \textit{merge sort},
além de ter uma das melhores complexidades $\Theta(n\log{}n)$, achamos 
que se enquadra perfeitamente no espirito do algoritmo, "Dividir para conquistar".

\subsubsection{Descrição de MergeSort}
Input: array, índice esquerdo, índice direito\\

Começando com o índice esquerdo a 0, e o índice direito a (\textit{tamanho do array}) - 1
\begin{itemize}
    \item Primeiro encontra-se o índice médio do Array e divide-se em dois. 
        ( meio = (esquerda + direita)/2 )
    \item Calcular o MergeSort(array, esquerda, meio) (a lista que fica à esquerda)
    \item Calcular o MergeSort(array, meio+1, direita) (a lista que fica à direita)
    \item No final junta os dois de forma ordenada.
\end{itemize}

\subsection{ Remover os Pontos Repetidos }
Já que neste contexto sabemos que o array está ordenado,
podemos fazer este algoritmo com 2 índices, um para percorrer
o array, o outro para percorrer o novo array onde todos os 
elementos são diferentes. Se o valor do array nos dois índices for
diferente então incrementa-se o segundo índice e atribui-se
o valor do primeiro índice no valor do segundo índice. Se o 
valor dos dois índices for igual, apenas incrementa-se o
primeiro índice.

\subsection{ Separar os Pontos em dois conjuntos }
No trabalho decidimos usar um array, para que a sua divisão a meio 
tivesse uma complexidade constante é tão simples como encontrar o 
índice a meio do array.

\subsection{ Encontrar as tangentes } \label{tangentes}
Na função que trata de encontrar os pontos que 
descrevem a tangente superior, começa-se com a linha
do maior vértice de $I(S_1)$ ao menor vértice de $I(S_2)$ (lexicograficamente).
A partir do vértice atual de $I(S_1)$ segue-se para o anterior, o mais 
longe possível, até que o vértice antecessor esteja estritamente à
esquerda da linha, de seguida fazemos o dual para o vértice de $I(S_2)$.
Repetindo este processo até não se conseguir mais ou encontrarmos
um dos pontos extremos (o ponto mais pequeno ou o ponto maior,
lexicograficamente).\\

Para encontrar a tangente inferior é o dual de encontrar
a superior.

\subsection{ Juntar os invólucros convexos } 
Liberta-se a memória dos vértices entre as duas ligações e de seguida 
conecta-se as duas ligações.

%--------------------------------------------

\section{Correção}
Depois de mostrarmos como é o algoritmo,
aqui vamos provar, porque é que o algoritmo faz o que diz que faz.
Tendo I como a nossa função descrita em cima.\\
\\
Queremos que no fim de execução do algoritmo, o poligono
retornado seja o invólucro convexo do conjunto dado 
no input.\\
Para isso vamos usar um Corolário dado nas aulas, descrito como:

\begin{center}
    "Seja $S \subseteq \mathbb{R}^2$ e $P$ um poligono de vértices 
    positivamente orientados $v_0,...,v_{n-1} \in S$ tal que cada
    vértice é estritamente convexo e $S \subseteq P$. Então $P$ é
    o invólucro convexo de S."
\end{center}
E a definição:

\begin{center}
    "$v_i$ diz-se estritamente convexo se $\mathcal{A}(v_{i-1},v_i,v_{i+1}) > 0$."
\end{center}

Seja o input $S$, conjunto de pontos, e o output $P$, poligono.\\
Caso Base ($S = \emptyset$):\par
\hfill\begin{minipage}{\dimexpr\textwidth-1cm}
    Trivial\\
\end{minipage}
Caso Base ($\#S = 1$):\par
\hfill\begin{minipage}{\dimexpr\textwidth-1cm}
    Seja $v_0 \in S$, então pela definição de $I$, $v_0 \in P$\\
\end{minipage}
Caso Indutivo:\par
\hfill\begin{minipage}{\dimexpr\textwidth-1cm}
    Seja $S = S_1 \cup S_2$ tal que todos os pontos de $S_1$
    sejam menores que os pontos de $S_2$ (lexicograficamente); 
    $P_1 = I(S_1)$ e $P_2 = I(S_2)$ em que $P_1$ e $P_2$ são 
    invólucros convexos por indução.\\\\
    
    \underline{Suponhamos que os pontos de $S$ são colineares.}\\
    Pelas definições de \ref{tangentes} vão encontrar as linhas 
    $\overline{p q}$ e $\overline{q p}$, respetivamente,
    tal que p é o ponto menor de S, e q o ponto maior de S. 
    Logo $P = \overline{p q}$, o invólucro convexo de S.\\\\

    \underline{Suponhamos que os pontos de $S$ não são colineares.}\\
    Agora para provar que $P$ é o invólucro convexo
    do conjunto $S$, basta provar que $P$, com os vértices 
    positivamente orientados $v_0,...,v_{n-1}$, $\mathbf{v_0,...,v_{n-1} \in S}$, 
    \textbf{todos os vértices são estritamente convexos} e 
    \textbf{todos os vértices estão no poligono}.\\
    
    
    
\end{minipage}

\subsection{$\mathbf{v_0,...,v_{n-1} \in S}$}

Hipoteses Indutivas:
\begin{enumerate}
    \item $pontos(P_1) \subseteq S_1$
    \item $pontos(P_2) \subseteq S_2$
\end{enumerate}
Como ao juntar os dois invólucros convexos $P_1$ e $P_2$
nunca acrescentamos pontos a $P$, além de pontos de $P_1$ e $P_2$, 
significa que os pontos de $P$ estão contidos na 
união dos pontos de $P_1$ e pontos de $P_2$.
Logo temos que $pontos(P) \subseteq pontos(P_1) \cup pontos(P_2)$.\\ \\
$pontos(P) \\
\subseteq \{$Provado acima$\}\\
pontos(P_1) \cup pontos(P_2) \\
\subseteq \{$Hipoteses de indução$\} \\
S_1 \cup S_2 \\
= \{$Definição de S$\}\\
S$


\subsection{Todos os vértices são estritamente convexos}

Hipoteses Indutivas:
\begin{enumerate}
    \item $< \forall p_i : p_i \in I(S_1) : \mathcal{A}(p_{i-1}, p_i, p_{i+1}) > 0 >$
    \item $< \forall p_i : p_i \in I(S_2) : \mathcal{A}(p_{i-1}, p_i, p_{i+1}) > 0 >$
\end{enumerate}


Pela descrição da "\ref{tangentes} Encontrar as tangentes"
\par

Quando não conseguirmos mais, significa que encontramos 2 vértices 
$p_i$ e $q_i$, tais que:\\
\\
$left(\overline{p_i q_i}, p_{i-1}) \wedge left(\overline{p_i q_i}, q_{i+1})$\\
$\equiv$ \{ definição de left \}\\
$\mathcal{A}(p_i, q_i, p_{i-1}) > 0 \wedge \mathcal{A}(p_i, q_i, q_{i+1}) > 0$\\
$\equiv$ \{ regra dada na aula \}\\
$\mathcal{A}(p_{i-1}, p_i, q_i) > 0 \wedge \mathcal{A}(p_i, q_i, q_{i+1}) > 0$\\

e arranjamos 2 vértices $p_i$ e $q_i$, tais que:\\
\\
$left(\overline{q_i p_i}, p_{i+1}) \wedge left(\overline{q_i p_i}, q_{i-1})$\\
$\equiv$ \{ definição de left \}\\
$\mathcal{A}(q_i, p_i, p_{i+1}) > 0 \wedge \mathcal{A}(q_i, p_i, q_{i-1}) > 0$\\
$\equiv$ \{ regra dada na aula \}\\
$\mathcal{A}(q_i, p_i, p_{i+1}) > 0 \wedge \mathcal{A}(q_{i-1}, q_i, p_i) > 0$\\

Com isto temos que os vértices encontrados que foram 
usados para juntar os dois invólucros são estritamente
convexos no $I(S)$. Como por indução sabemos que os 
outros vértices são estritamente convexos, significa
que todos os vértices em $I(S)$ são estritamente convexos.


\subsection{Todos os vértices estão no poligono}
Hipoteses Indutivas:
\begin{enumerate}
    \item $S_1 \subseteq P_1$ 
    \item $S_2 \subseteq P_2$.
\end{enumerate}

Seja $p_i$ o menor ponto da interseção da tangente 
superior com $P_1$, e $q_i$ o maior ponto da interseção com $P_2$.
Analogamente $p_j$ o menor ponto da interseção da tangente 
inferior com $P_1$ e $q_j$ o maior ponto da interseção com $P_2$.

Depois de se juntar $P_1$ e $P_2$ cria-se um invólucro convexo,
onde se pode dividir em 3 partes.

\begin{itemize}
    \item A: Todos os pontos de $S_1$ à esquerda de $p_jp_i$
    \item B: Poligono formado pelos pontos \{$p_i,p_j,q_j,q_i$\}
    \item C: Todos os pontos de $S_2$ à direita de $q_jq_i$
\end{itemize}
Ou seja $P = A \cup B \cup C$.\\
Se $S_1 \subseteq A \cup B$ e $S_2 \subseteq B \cup C$,
então podemos concluir que $S_1 \cup S_2 = S \subseteq P$.\\
Por isso queremos provar que:
\begin{enumerate}
    \item $S_1 \subseteq A \cup B$
    \item $S_2 \subseteq B \cup C$
\end{enumerate}
Seja $A'$ são os pontos de $S_1$ à direita de $p_ip_j$,
temos que $S_1 = A \cup A'$. 
\newline
Queremos provar que $A' \subseteq B$.\\
Por definição B contém os pontos à direita de $p_jp_i$ e à esquerda
de $q_jq_i$ e como os pontos são ordenados e divididos em $P_1$ e $P_2$
temos a garantia que a linha $q_jq_i$ está à direita de $p_jp_i$.
Uma vez que $q_ip_i$ é uma tangente superior e $p_jq_j$ é uma tangente
inferior, significa que todos os pontos de $A'$ estão à esquerda de
ambas essas tangentes. Logo $A' \subseteq B$.
Concluindo que $S_1 = A \cup A' \subseteq A \cup B$\\ \\
Analogamete para $S_2 \subseteq B \cup C$
\hfill $\blacksquare$

\subsection{Proposição Tangente}
Seja 
    I um invólucro convexo e 
    l uma linha que interseta num vértice de I, $p_i$,
no sentido contrário dos ponteiros do relógio.
Seja $p_{i-1}$ o vértice anterior a $p_i$ e $p_{i+1}$ o vértice posterior a $p_i$.
Temos que:
$$leftOn(l, p_{i-1}) \wedge leftOn(l, p_{i+1}) \Rightarrow
    <\forall p : p \in I : leftOn(l, p)>$$

\subsubsection{Demonstração}
Suponhamos que $leftOn(l, p_{i-1}) \wedge leftOn(l, p_{i+1})$ 
e que existe um $p \in I$ tal que $right(l, p)$
isso significa que a linha $\overline{p\ p_i}$ não está no invólucro convexo.
Que é uma contradição, ou seja, $<\forall p : p \in I : leftOn(l, p)>$

\hfill $\blacksquare$
%--------------------------------------------


\section{Complexidade}
A nossa implementação não foi exatamente igual ao algoritmo original,
apesar de não alterar na conta da complexidade assintoticamente.
Por isso vamos avaliar a correção da nossa implementação, e quando 
achamos pertinente, vamos fazer a ressalva, mensionando as diferenças 
em relação ao algoritmo original.

\begin{itemize}
    \item Ordenar os pontos pela cordenada x, tem Complexidade $\Theta(n\log{}n)$
    \item Separar os pontos em dois conjuntos A e B, onde A contém os pontos da esquerda e B os da direita.
    \item Calcular o invólucro convexo de A, $\mathcal{A}$ = I(A) e o de B, $\mathcal{B}$ = I(B) recursivamente
    \item No final juntar $\mathcal{A}$ e $\mathcal{B}$, calculando o invólucro convexo de A $\cup$ B.
\end{itemize}

\subsection{Ordenar pontos}
Pela complexidade já conhecida do \textit{merge sort}, 
temos que a complexidade de Ordenar pontos é:
$$\Theta(n\log{}n)$$

\subsection{Remover pontos repetidos}
Como os pontos são recebidos ordenados lexicograficamente,
significa que podemos eliminar os pontos iguais com apenas
uma simples passagem pelo array, tendo em conta apenas o 
índice do novo array e o indice do array antigo, logo a 
complexidade é:
$$\mathcal{O}(n)$$

\subsection{Separar conjuntos de pontos}
Para isso basta encontrar o meio do array,
e como temos o tamanho do array é simples, faz-se apenas com uma operação,
divisão por 2, o que torna a complexidade desta operação:
$$\mathcal{O}(1)$$

\subsection{Encontrar tanjentes}
Para encontrar a tangente inferior(superior),
o pior caso é quando encontra os pontos limite
(o ponto mais à esquerda e mais à direita do conjunto de pontos) 
simultaneamente na tangente inferior(superior), 
para isso tem que percorrer todos os pontos,
o que torna a sua complexidade: 
$$\mathcal{O}(n)$$

\subsection{Juntar os invólucros convexos}
Como para os poligonos usamos uma lista duplamente ligada e
para juntar recebemos os 4 pontos a juntar, é só preciso 
"ligar" os pontos com 8 apontadores, ou seja, a complexidade é: 
$$\mathcal{O}(1)$$
(Nota: na prática libertamos a memória que estava 
a meio das tangentes fazendo a complexidade linear
mas poderiamos ter guardado num conjunto de poligonos 
a libertar, tornando a complexidade constante e no final 
teriamos de libertar, não mudando a complexidade total)

\subsection{Calcular invólucro}
$$
    \mathcal{T}(n) = 
        \begin{cases}
            \mathcal{O}(1)                      & n = 0\\
            \mathcal{O}(1)                      & n = 1\\
            \mathcal{O}(n) + 2*\mathcal{T}(n/2) & n > 1
        \end{cases}
$$

$$
    \mathcal{T}(n) = \sum_{i=0}^{\lfloor\log_{2}n\rfloor} 2^i*\mathcal{O}(n/2^i)
    = \sum_{i=0}^{\lfloor\log_{2}n\rfloor} \mathcal{O}(n)
    = (\lfloor\log_{2}n\rfloor + 1)*\mathcal{O}(n)
    = \mathcal{O}(n\log{n})
$$
$$
    \mathcal{O}(n\log{n})
$$

%--------------------------------------------

\section{Conclusão}
Com o desenvolvimento deste algoritmo, achamos uma boa ideia 
para questões de debugging e apresentação criarmos uma ferramenta
gráfica, \textit{visualize} para introdução de pontos e \textit{animate}
para visualizar as execuções do programa desenvolvido, passo a passo.\\
Constatamos que a complexidade referida no documento partilhado 
pelo professor (Computational geometry in C) referia erradamente 
a complexidade deste algoritmo, sendo não $\mathcal{O}(n^2)$, mas sim
$\mathcal{O}(n\log{n})$ como mostramos neste trabalho. Por nossa
surpresa quando experimentamos medir o tempo usando uma variadade de \\

\begin{itemize}
    \item O documento estava errado da complexidade.
    \item A parte de encontrar os limites pode ser mais eficiente, 
        mas a complexidade não mudaria, porque temos de ordenar.
    \item Paralelismo, to be done
    \item ...
\end{itemize}


\end{document}
